\documentclass[12pt]{article}
\usepackage{amsmath}
\DeclareMathOperator{\Tr}{Tr}
\DeclareMathOperator{\Aut}{Aut}
\DeclareMathOperator{\ad}{ad}

\usepackage{amsmath,amsthm,amssymb}
\usepackage{mathtext}
\usepackage{mathtools}
\usepackage[T1,T2A]{fontenc}
\usepackage[utf8x]{inputenc}
\usepackage[english,russian]{babel}
\usepackage{fancyhdr}
\renewcommand{\leq}{\leqslant}
\renewcommand{\geq}{\geqslant}
\renewcommand{\epsilon}{\varepsilon}


\addtolength{\topmargin}{-5mm}
\addtolength{\textheight}{40mm}
\addtolength{\oddsidemargin}{-25mm}
\addtolength{\textwidth}{45mm}

\setlength{\headheight}{15pt}
\pagestyle{fancy} 


\rhead{Симметрия и теория групп. Весна 2019}



\def\rad{\operatorname{\sf rad}}
\def\Alt{\operatorname{\sf Alt}}
\def\Sym{\operatorname{\sf Sym}}
\def\Id{\operatorname{\sf Id}}
\def\rk{\operatorname{\sf rk}}
\def\Hom{\operatorname{Hom}}
\def\Vol{\operatorname{Vol}}
\def\Lie{\operatorname{Lie}}
\def\Map{\operatorname{Map}}
\def\Aut{\operatorname{Aut}}
\newcommand{\End}{\operatorname{End}}
\newcommand{\Mat}{\operatorname{Mat}}
\newcommand{\diam}{\operatorname{\sf diam}}

\def\Z{{\mathbb Z}}
\def\R{{\mathbb R}}
\def\C{{\mathbb C}}
\def\Q{{\mathbb Q}}
\def\N{{\mathbb N}}

\newtheoremstyle{Q}% <name>
{3pt}% <Space above>
{3pt}% <Space below>
{}% <Body font>
{}% <Indent amount>
{\bfseries}% <Theorem head font>
{:}% <Punctuation after theorem head>
{.5em}% <Space after theorem headi>
{}
\theoremstyle{Q}

\newtheorem{Problem}{Задача}[section]
\newcommand{\listok}[2]{%
	\setcounter{page}{1}
	\lhead{ \scriptsize #2 }
	\section*{#2}
	\refstepcounter{section}
	\setcounter{section}{#1}
}
\theoremstyle{generic}

\newtheorem{Definition}{Определение}[section]
\DeclareMathOperator{\s}{sign}
\def\p{\begin{Problem}}
\def\ep{\end{Problem}}
\def\df{\begin{Definition}}
\def\edf{\end{Definition}}



\renewcommand{\leq}{\leqslant}
\renewcommand{\geq}{\geqslant}
\renewcommand{\epsilon}{\varepsilon}

\begin{document}
	%%%%%%%%%%%%%%%%%%%%%%%%%%%%%%%%%%%%%%%%%%%%%%%%
	
	Для получения оценки 'отлично' за курс необходимо сдать не менее 9 задач из каждого листочка, оценки 'хорошо' - 7 задач, оценки 'удовлетворительно' - 5 задач. 
	
	%%%%%%%%%%%%%%%%%%%%%%%%%%%%%%%%%%%%%%%%%%%%%%%%%%%%%%%%%%%%
	\listok{3}{Группы SU(2), SO(3)}
	\p
	Докажите равенство:
	$$ e^{xA}Be^{-xA} = \sum \frac{x^k}{x!} \ad_{A}^k (B),$$
	где оператор $\ad$ определяется как: $\ad_a(b) = [a,b]$. 
	\ep
	\p Покажите, что центр группы $SO(n)$ либо тривиален, либо имеет порядок 2. Какой центр у $SU(n)$?.
	\ep


	\p
	Введем на $SU(2)$ следующую метрику:
	$$ds^2 = \Tr  (dg\cdot dg^\dagger)$$
	Покажите что эта метрика совпадает со стандартной метрикой на соотвествующей сфере.
	\ep
	\p
	Покажите, что любой элемент $O \in SO(3)$ имеет собственный вектор с собственным значением $1$.
	\ep
	\p
	Покажите, что алгебра Ли $\mathfrak{so}(3)$ - векторное пространство антисимметричных матриц. 
	\ep
	Рассмотрим пространство бесседловых эрмитовых матриц:
	$$H = \{ h \in \mathbb{C}^{2x2} | \Tr(h)=0, h=h^\dagger  \}$$
	\p
	Покажите, что матрицы Паули($\sigma_i$) - базис $H$.
	Покажите, что $\forall U \in SU(2), \forall m \in H: \ U^\dagger m U = n \in H$
	\ep
	\p 
	Пусть $m=m_i \sigma_i \in H$. Докажите, что отображение:
	$$\omega: SU(2) \to \Aut(\mathbb{R}^3)$$
	$$U \mapsto \omega(U)$$
	такое что $n_i = \omega(U)_{ij} m_j$ задаётся формулой $$\omega(U)_{ij} = \frac 1 2 \Tr(\sigma_i U^\dagger \sigma_j U) $$ Покажите что это гомоморфизм.
	\ep
	\p
	Покажите, что $\omega(U) \in SO(3)$.
	\ep
	\p 	
	Докажите, что $SO(3) \simeq SU(2) / Z_2$
	\ep


\end{document}