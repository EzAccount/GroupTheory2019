\documentclass[12pt]{article}


\usepackage{amsmath,amsthm,amssymb}
\usepackage{mathtext}
\usepackage{mathtools}
\usepackage[T1,T2A]{fontenc}
\usepackage[utf8x]{inputenc}
\usepackage[english,russian]{babel}
\usepackage{fancyhdr}
\renewcommand{\leq}{\leqslant}
\renewcommand{\geq}{\geqslant}
\renewcommand{\epsilon}{\varepsilon}


\addtolength{\topmargin}{-5mm}
\addtolength{\textheight}{40mm}
\addtolength{\oddsidemargin}{-25mm}
\addtolength{\textwidth}{45mm}

\setlength{\headheight}{15pt}
\pagestyle{fancy} 


\rhead{Симметрия и теория групп. Весна 2019}



\def\rad{\operatorname{\sf rad}}
\def\Alt{\operatorname{\sf Alt}}
\def\Sym{\operatorname{\sf Sym}}
\def\Id{\operatorname{\sf Id}}
\def\rk{\operatorname{\sf rk}}
\def\Hom{\operatorname{Hom}}
\def\Vol{\operatorname{Vol}}
\def\Lie{\operatorname{Lie}}
\def\Map{\operatorname{Map}}
\def\Aut{\operatorname{Aut}}
\newcommand{\End}{\operatorname{End}}
\newcommand{\Mat}{\operatorname{Mat}}
\newcommand{\diam}{\operatorname{\sf diam}}

\def\Z{{\mathbb Z}}
\def\R{{\mathbb R}}
\def\C{{\mathbb C}}
\def\Q{{\mathbb Q}}
\def\N{{\mathbb N}}

\newtheoremstyle{Q}% <name>
{3pt}% <Space above>
{3pt}% <Space below>
{}% <Body font>
{}% <Indent amount>
{\bfseries}% <Theorem head font>
{:}% <Punctuation after theorem head>
{.5em}% <Space after theorem headi>
{}
\theoremstyle{Q}

\newtheorem{Problem}{Задача}[section]
\newcommand{\listok}[2]{%
	\setcounter{page}{1}
	\lhead{ \scriptsize #2 }
	\section*{#2}
	\refstepcounter{section}
	\setcounter{section}{#1}
}
\theoremstyle{generic}

\newtheorem{Definition}{Определение}[section]
\DeclareMathOperator{\s}{sign}
\def\p{\begin{Problem}}
\def\ep{\end{Problem}}
\def\df{\begin{Definition}}
\def\edf{\end{Definition}}



\begin{document}
	%%%%%%%%%%%%%%%%%%%%%%%%%%%%%%%%%%%%%%%%%%%%%%%%
	
	Для получения оценки 'отлично' за курс необходимо сдать не менее 9 задач из каждого листочка, оценки 'хорошо' - 7 задач, оценки 'удовлетворительно' - 5 задач. 
	
	%%%%%%%%%%%%%%%%%%%%%%%%%%%%%%%%%%%%%%%%%%%%%%%%%%%%%%%%%%%%
	\listok{1}{Понятие группы, морфизмы. Группа перестановок}
	\p
	Покажите, что любая перестановка представима как произведение перестановок вида $(1,i)$; как произведение перестановок вида $(i,i+1)$.
	\ep
	\df
	$\s(\pi)=(-1)^t$, где t - количество транспозиций в разложении $\pi$ 
	\edf
	\p
	Покажите независимость определения $\s$ от разложения. Сформулируйте это определение в терминах количества инверсий в перестановке.  
	\ep
	\p
	Докажите, что отображение $\s: S_n \to \Z_2$ - гомоморфизм
	\ep


	\p 
	Установите следующие изоморфизмы:
	\begin{align*}
	S_3 &\simeq \text{[Группа движений треугольника]} \\
	S_4 &\simeq \text{[Группа вращений куба]}\\
	S_4 \times S_2 &\simeq{\text[Группа движений куба]}
	\end{align*}
	\ep
	
	\p Доказать, что группа из 6 элементов либо абелева, либо изоморфна $S_3$
	\ep
	
	\df Автоморфизмом группы называется изоморфизм $G \to G$. Группа автоморфизмов группы $G$ обозначается $\Aut G$.
	\edf

	\p Докажите, что
	$\Aut(\Z_p) \simeq (\Z_p)$
	\ep
	
	\p Покажите, что любая бесконечная группа содержит нетривиальную подгруппу.
	\ep
	
	\p Докажите, что любая группа порядка 8 имеет вид $\{1,a, a^2, a^3, b, ab, a^2 b, a^3 b\}$. 
	\ep
	
	\p 
	Каких перестановок в $S_n$ больше - чётных или нечётных?
	\ep

	\p
	Введём на множестве $G$ бинарную операцию $/$ следующим образом:
	$$
		G \times G \to G: (g,h) \mapsto g/h
	$$
	$$
	\forall f,g,h \in G: (f/h)/(g/h) = f/g
	$$
	$$
	\forall g,h \in G, \exists x \in G: g/x=h
	$$
	Покажите, что $G$ - группа относительно умножения $gh = g/ ((h/h)/h)$.
	\ep
	
\end{document}